%% template.tex
%% from
%% bare_conf.tex
%% V1.4b
%% 2015/08/26
%% by Michael Shell
%% See:
%% http://www.michaelshell.org/
%% for current contact information.
%%
%% This is a skeleton file demonstrating the use of IEEEtran.cls
%% (requires IEEEtran.cls version 1.8b or later) with an IEEE
%% conference paper.
%%
%% Support sites:
%% http://www.michaelshell.org/tex/ieeetran/
%% http://www.ctan.org/pkg/ieeetran
%% and
%% http://www.ieee.org/

%%*************************************************************************
%% Legal Notice:
%% This code is offered as-is without any warranty either expressed or
%% implied; without even the implied warranty of MERCHANTABILITY or
%% FITNESS FOR A PARTICULAR PURPOSE!
%% User assumes all risk.
%% In no event shall the IEEE or any contributor to this code be liable for
%% any damages or losses, including, but not limited to, incidental,
%% consequential, or any other damages, resulting from the use or misuse
%% of any information contained here.
%%
%% All comments are the opinions of their respective authors and are not
%% necessarily endorsed by the IEEE.
%%
%% This work is distributed under the LaTeX Project Public License (LPPL)
%% ( http://www.latex-project.org/ ) version 1.3, and may be freely used,
%% distributed and modified. A copy of the LPPL, version 1.3, is included
%% in the base LaTeX documentation of all distributions of LaTeX released
%% 2003/12/01 or later.
%% Retain all contribution notices and credits.
%% ** Modified files should be clearly indicated as such, including  **
%% ** renaming them and changing author support contact information. **
%%*************************************************************************


% *** Authors should verify (and, if needed, correct) their LaTeX system  ***
% *** with the testflow diagnostic prior to trusting their LaTeX platform ***
% *** with production work. The IEEE's font choices and paper sizes can   ***
% *** trigger bugs that do not appear when using other class files.       ***                          ***
% The testflow support page is at:
% http://www.michaelshell.org/tex/testflow/

\documentclass[conference,final,]{IEEEtran}
% Some Computer Society conferences also require the compsoc mode option,
% but others use the standard conference format.
%
% If IEEEtran.cls has not been installed into the LaTeX system files,
% manually specify the path to it like:
% \documentclass[conference]{../sty/IEEEtran}





% Some very useful LaTeX packages include:
% (uncomment the ones you want to load)


% *** MISC UTILITY PACKAGES ***
%
%\usepackage{ifpdf}
% Heiko Oberdiek's ifpdf.sty is very useful if you need conditional
% compilation based on whether the output is pdf or dvi.
% usage:
% \ifpdf
%   % pdf code
% \else
%   % dvi code
% \fi
% The latest version of ifpdf.sty can be obtained from:
% http://www.ctan.org/pkg/ifpdf
% Also, note that IEEEtran.cls V1.7 and later provides a builtin
% \ifCLASSINFOpdf conditional that works the same way.
% When switching from latex to pdflatex and vice-versa, the compiler may
% have to be run twice to clear warning/error messages.






% *** CITATION PACKAGES ***
%
%\usepackage{cite}
% cite.sty was written by Donald Arseneau
% V1.6 and later of IEEEtran pre-defines the format of the cite.sty package
% \cite{} output to follow that of the IEEE. Loading the cite package will
% result in citation numbers being automatically sorted and properly
% "compressed/ranged". e.g., [1], [9], [2], [7], [5], [6] without using
% cite.sty will become [1], [2], [5]--[7], [9] using cite.sty. cite.sty's
% \cite will automatically add leading space, if needed. Use cite.sty's
% noadjust option (cite.sty V3.8 and later) if you want to turn this off
% such as if a citation ever needs to be enclosed in parenthesis.
% cite.sty is already installed on most LaTeX systems. Be sure and use
% version 5.0 (2009-03-20) and later if using hyperref.sty.
% The latest version can be obtained at:
% http://www.ctan.org/pkg/cite
% The documentation is contained in the cite.sty file itself.






% *** GRAPHICS RELATED PACKAGES ***
%
\ifCLASSINFOpdf
  % \usepackage[pdftex]{graphicx}
  % declare the path(s) where your graphic files are
  % \graphicspath{{../pdf/}{../jpeg/}}
  % and their extensions so you won't have to specify these with
  % every instance of \includegraphics
  % \DeclareGraphicsExtensions{.pdf,.jpeg,.png}
\else
  % or other class option (dvipsone, dvipdf, if not using dvips). graphicx
  % will default to the driver specified in the system graphics.cfg if no
  % driver is specified.
  % \usepackage[dvips]{graphicx}
  % declare the path(s) where your graphic files are
  % \graphicspath{{../eps/}}
  % and their extensions so you won't have to specify these with
  % every instance of \includegraphics
  % \DeclareGraphicsExtensions{.eps}
\fi
% graphicx was written by David Carlisle and Sebastian Rahtz. It is
% required if you want graphics, photos, etc. graphicx.sty is already
% installed on most LaTeX systems. The latest version and documentation
% can be obtained at:
% http://www.ctan.org/pkg/graphicx
% Another good source of documentation is "Using Imported Graphics in
% LaTeX2e" by Keith Reckdahl which can be found at:
% http://www.ctan.org/pkg/epslatex
%
% latex, and pdflatex in dvi mode, support graphics in encapsulated
% postscript (.eps) format. pdflatex in pdf mode supports graphics
% in .pdf, .jpeg, .png and .mps (metapost) formats. Users should ensure
% that all non-photo figures use a vector format (.eps, .pdf, .mps) and
% not a bitmapped formats (.jpeg, .png). The IEEE frowns on bitmapped formats
% which can result in "jaggedy"/blurry rendering of lines and letters as
% well as large increases in file sizes.
%
% You can find documentation about the pdfTeX application at:
% http://www.tug.org/applications/pdftex





% *** MATH PACKAGES ***
%
%\usepackage{amsmath}
% A popular package from the American Mathematical Society that provides
% many useful and powerful commands for dealing with mathematics.
%
% Note that the amsmath package sets \interdisplaylinepenalty to 10000
% thus preventing page breaks from occurring within multiline equations. Use:
%\interdisplaylinepenalty=2500
% after loading amsmath to restore such page breaks as IEEEtran.cls normally
% does. amsmath.sty is already installed on most LaTeX systems. The latest
% version and documentation can be obtained at:
% http://www.ctan.org/pkg/amsmath





% *** SPECIALIZED LIST PACKAGES ***
%
%\usepackage{algorithmic}
% algorithmic.sty was written by Peter Williams and Rogerio Brito.
% This package provides an algorithmic environment fo describing algorithms.
% You can use the algorithmic environment in-text or within a figure
% environment to provide for a floating algorithm. Do NOT use the algorithm
% floating environment provided by algorithm.sty (by the same authors) or
% algorithm2e.sty (by Christophe Fiorio) as the IEEE does not use dedicated
% algorithm float types and packages that provide these will not provide
% correct IEEE style captions. The latest version and documentation of
% algorithmic.sty can be obtained at:
% http://www.ctan.org/pkg/algorithms
% Also of interest may be the (relatively newer and more customizable)
% algorithmicx.sty package by Szasz Janos:
% http://www.ctan.org/pkg/algorithmicx




% *** ALIGNMENT PACKAGES ***
%
%\usepackage{array}
% Frank Mittelbach's and David Carlisle's array.sty patches and improves
% the standard LaTeX2e array and tabular environments to provide better
% appearance and additional user controls. As the default LaTeX2e table
% generation code is lacking to the point of almost being broken with
% respect to the quality of the end results, all users are strongly
% advised to use an enhanced (at the very least that provided by array.sty)
% set of table tools. array.sty is already installed on most systems. The
% latest version and documentation can be obtained at:
% http://www.ctan.org/pkg/array


% IEEEtran contains the IEEEeqnarray family of commands that can be used to
% generate multiline equations as well as matrices, tables, etc., of high
% quality.




% *** SUBFIGURE PACKAGES ***
%\ifCLASSOPTIONcompsoc
%  \usepackage[caption=false,font=normalsize,labelfont=sf,textfont=sf]{subfig}
%\else
%  \usepackage[caption=false,font=footnotesize]{subfig}
%\fi
% subfig.sty, written by Steven Douglas Cochran, is the modern replacement
% for subfigure.sty, the latter of which is no longer maintained and is
% incompatible with some LaTeX packages including fixltx2e. However,
% subfig.sty requires and automatically loads Axel Sommerfeldt's caption.sty
% which will override IEEEtran.cls' handling of captions and this will result
% in non-IEEE style figure/table captions. To prevent this problem, be sure
% and invoke subfig.sty's "caption=false" package option (available since
% subfig.sty version 1.3, 2005/06/28) as this is will preserve IEEEtran.cls
% handling of captions.
% Note that the Computer Society format requires a larger sans serif font
% than the serif footnote size font used in traditional IEEE formatting
% and thus the need to invoke different subfig.sty package options depending
% on whether compsoc mode has been enabled.
%
% The latest version and documentation of subfig.sty can be obtained at:
% http://www.ctan.org/pkg/subfig




% *** FLOAT PACKAGES ***
%

%\usepackage{fixltx2e}
% fixltx2e, the successor to the earlier fix2col.sty, was written by
% Frank Mittelbach and David Carlisle. This package corrects a few problems
% in the LaTeX2e kernel, the most notable of which is that in current
% LaTeX2e releases, the ordering of single and double column floats is not
% guaranteed to be preserved. Thus, an unpatched LaTeX2e can allow a
% single column figure to be placed prior to an earlier double column
% figure.
% Be aware that LaTeX2e kernels dated 2015 and later have fixltx2e.sty's
% corrections already built into the system in which case a warning will
% be issued if an attempt is made to load fixltx2e.sty as it is no longer
% needed.
% The latest version and documentation can be found at:
% http://www.ctan.org/pkg/fixltx2e


%\usepackage{stfloats}
% stfloats.sty was written by Sigitas Tolusis. This package gives LaTeX2e
% the ability to do double column floats at the bottom of the page as well
% as the top. (e.g., "\begin{figure*}[!b]" is not normally possible in
% LaTeX2e). It also provides a command:
%\fnbelowfloat
% to enable the placement of footnotes below bottom floats (the standard
% LaTeX2e kernel puts them above bottom floats). This is an invasive package
% which rewrites many portions of the LaTeX2e float routines. It may not work
% with other packages that modify the LaTeX2e float routines. The latest
% version and documentation can be obtained at:
% http://www.ctan.org/pkg/stfloats
% Do not use the stfloats baselinefloat ability as the IEEE does not allow
% \baselineskip to stretch. Authors submitting work to the IEEE should note
% that the IEEE rarely uses double column equations and that authors should try
% to avoid such use. Do not be tempted to use the cuted.sty or midfloat.sty
% packages (also by Sigitas Tolusis) as the IEEE does not format its papers in
% such ways.
% Do not attempt to use stfloats with fixltx2e as they are incompatible.
% Instead, use Morten Hogholm'a dblfloatfix which combines the features
% of both fixltx2e and stfloats:
%
% \usepackage{dblfloatfix}
% The latest version can be found at:
% http://www.ctan.org/pkg/dblfloatfix




% *** PDF, URL AND HYPERLINK PACKAGES ***
%
%\usepackage{url}
% url.sty was written by Donald Arseneau. It provides better support for
% handling and breaking URLs. url.sty is already installed on most LaTeX
% systems. The latest version and documentation can be obtained at:
% http://www.ctan.org/pkg/url
% Basically, \url{my_url_here}.




% *** Do not adjust lengths that control margins, column widths, etc. ***
% *** Do not use packages that alter fonts (such as pslatex).         ***
% There should be no need to do such things with IEEEtran.cls V1.6 and later.
% (Unless specifically asked to do so by the journal or conference you plan
% to submit to, of course. )



%% BEGIN MY ADDITIONS %%


\usepackage{graphicx}
% We will generate all images so they have a width \maxwidth. This means
% that they will get their normal width if they fit onto the page, but
% are scaled down if they would overflow the margins.
\makeatletter
\def\maxwidth{\ifdim\Gin@nat@width>\linewidth\linewidth
\else\Gin@nat@width\fi}
\makeatother
\let\Oldincludegraphics\includegraphics
\renewcommand{\includegraphics}[1]{\Oldincludegraphics[width=\maxwidth]{#1}}

\usepackage[unicode=true]{hyperref}
\usepackage[spanish]{babel}
\usepackage[utf8]{inputenc} 
\usepackage[T1]{fontenc}
\usepackage{lmodern}
\usepackage{natbib}
\bibliographystyle{apalike}

\hypersetup{
            pdftitle={Análisis de variables climáticas del páramo la Rusia perteneciente al departamento de Boyacá},
            pdfkeywords={Páramo, Relaciónes climáticas, Temperatura, Velocidad del viento, Precipitación},
            pdfborder={0 0 0},
            breaklinks=true}
\urlstyle{same}  % don't use monospace font for urls

% Pandoc toggle for numbering sections (defaults to be off)
\setcounter{secnumdepth}{0}

% Pandoc syntax highlighting


% Pandoc header

\providecommand{\tightlist}{%
  \setlength{\itemsep}{0pt}\setlength{\parskip}{0pt}}

%% END MY ADDITIONS %%


\hyphenation{op-tical net-works semi-conduc-tor}

\begin{document}
%
% paper title
% Titles are generally capitalized except for words such as a, an, and, as,
% at, but, by, for, in, nor, of, on, or, the, to and up, which are usually
% not capitalized unless they are the first or last word of the title.
% Linebreaks \\ can be used within to get better formatting as desired.
% Do not put math or special symbols in the title.
\title{Análisis de variables climáticas del páramo la Rusia perteneciente al
departamento de Boyacá}

% author names and affiliations
% use a multiple column layout for up to three different
% affiliations

\author{

%% ---- classic IEEETrans wide authors' list ----------------
 % -- end affiliation.wide
%% ----------------------------------------------------------



%% ---- classic IEEETrans one column per institution --------
 %% -- beg if/affiliation.institution-columnar
\IEEEauthorblockN{
  %% -- beg for/affiliation.institution.author
Camilo Andres Cruz Sanchez %% -- end for/affiliation.institution.author
}
\IEEEauthorblockA{Departamento de Ciencias Forestales\\
Universidad Nacional de Colombia\\
Medellín, Antioquia
\\cacruzs@unal.edu.co
}
\and
\IEEEauthorblockN{
  %% -- beg for/affiliation.institution.author
Natali Andrea Lopez Toro %% -- end for/affiliation.institution.author
}
\IEEEauthorblockA{Area Curricular de Medio Ambiente\\
Universidad Nacional de Colombia\\
Medellín, Antioquia
\\naalopezto@unal.edu.co
}
\and
\IEEEauthorblockN{
  %% -- beg for/affiliation.institution.author
Juan David Leon Torres %% -- end for/affiliation.institution.author
}
\IEEEauthorblockA{Departamento de Ciencias Forestales\\
Universidad Nacional de Colombia\\
Medellín, Antioquia
\\judleonto@unal.edu.co
}
\and
\IEEEauthorblockN{
  %% -- beg for/affiliation.institution.author
Cristian Camilo Gañan Tapasco %% -- end for/affiliation.institution.author
}
\IEEEauthorblockA{Departamento de Ciencias Forestales\\
Universidad Nacional de Colombia\\
Medellín, Antioquia
\\ccganant@unal.edu.co
}
 %% -- end for/affiliation.institution
 %% -- end if/affiliation.institution-columnar
%% ----------------------------------------------------------





%% ---- one column per author, classic/default IEEETrans ----
 %% -- end if/affiliation.institution-columnar
%% ----------------------------------------------------------

}

% conference papers do not typically use \thanks and this command
% is locked out in conference mode. If really needed, such as for
% the acknowledgment of grants, issue a \IEEEoverridecommandlockouts
% after \documentclass

% for over three affiliations, or if they all won't fit within the width
% of the page, use this alternative format:
%
%\author{\IEEEauthorblockN{Michael Shell\IEEEauthorrefmark{1},
%Homer Simpson\IEEEauthorrefmark{2},
%James Kirk\IEEEauthorrefmark{3},
%Montgomery Scott\IEEEauthorrefmark{3} and
%Eldon Tyrell\IEEEauthorrefmark{4}}
%\IEEEauthorblockA{\IEEEauthorrefmark{1}School of Electrical and Computer Engineering\\
%Georgia Institute of Technology,
%Atlanta, Georgia 30332--0250\\ Email: see http://www.michaelshell.org/contact.html}
%\IEEEauthorblockA{\IEEEauthorrefmark{2}Twentieth Century Fox, Springfield, USA\\
%Email: homer@thesimpsons.com}
%\IEEEauthorblockA{\IEEEauthorrefmark{3}Starfleet Academy, San Francisco, California 96678-2391\\
%Telephone: (800) 555--1212, Fax: (888) 555--1212}
%\IEEEauthorblockA{\IEEEauthorrefmark{4}Tyrell Inc., 123 Replicant Street, Los Angeles, California 90210--4321}}




% use for special paper notices
%\IEEEspecialpapernotice{(Invited Paper)}




% make the title area
\maketitle

% As a general rule, do not put math, special symbols or citations
% in the abstract
\begin{abstract}
Los páramos son ecosistemas muy complejos e importantes por el papel que
juegan en la regulación y conservación del recurso hídrico por lo cual
se hace necesario entender el comportamiento de las variables climáticas
que se presenta en ellos, es por esto que se realiza un estudio en el
páramo de La Rusia donde se toman datos de precipitación, humedad
relativa, temperatura, radiación solar y velocidad del viento para 6
meses del año 2019 a través de una estación climatológica que hace parte
de un complejo de estaciones climáticas que ha venido instalando el
equipo de trabajo del Dr.~Mark Mulligan de Kings College y dos sensores
HOBO con los cuales se midieron temperatura y humedad relativa a través
del páramo durante cuatro días del mes de abril. Datos con los cuales se
hicieron relaciones entre las diferentes variables climáticas por medio
del software R los cuales mostraron correlaciones positivas para la
humedad relativa y precipitación(0.38) y correlaciones negativas entre
la radiación y humedad relativa(-0.81). Para los datos de los sensores
se realizó un modelo kriging ordinario de primer orden y transformación
logarítmica por medio del software Arcgis, el cual mostró una
disminución en la temperatura a medida que se aumentaba la altitud, pero
no con una relación lineal sino que era fluctuante.
\end{abstract}

% keywords
\begin{IEEEkeywords}
Páramo; Relaciónes climáticas; Temperatura; Velocidad del viento; Precipitación
\end{IEEEkeywords}

% use for special paper notices



% make the title area
\maketitle

% no keywords

% For peer review papers, you can put extra information on the cover
% page as needed:
% \ifCLASSOPTIONpeerreview
% \begin{center} \bfseries EDICS Category: 3-BBND \end{center}
% \fi
%
% For peerreview papers, this IEEEtran command inserts a page break and
% creates the second title. It will be ignored for other modes.
\IEEEpeerreviewmaketitle


\hypertarget{introducciuxf3n}{%
\section{Introducción}\label{introducciuxf3n}}

El páramo es uno de los ecosistemas más importantes para la captura de
agua, este se encuentra presente en un \(99 \%\) en la Cordillera de los
Andes, en la Sierra Nevada de Santa Marta y Costa Rica, pero también
existen Páramos en África, Indonesia y Papua Nueva Guinea
\cite{cabrera}. Es por esto que los páramos ubicados en la Cordillera de
los Andes han sido definidos como extensas zonas en la cima de la
cordilleras, entre el bosque andino y el límite inferior de las nieves
perpetuas \cite{cabrera}, haciendo privilegiados a los pocos países en
el mundo que cuentan con este tipo de ecosistema por la riqueza acuífera
que ellos representan. Para el caso de Colombia en el que se encuentran
el \(49 \%\) de los páramos del mundo, ocupando el \(1.7 \%\) del
territorio nacional con aproximadamente \(34\) páramos \cite{cabrera},de
estos según el ministerio de ambiente, el departamento de Boyacá cuenta
con el \(18.7 \%\) del total nacional. Conteniendo en \(16\) municipios
de este departamento se encuentra el páramo de La Rusia, en el cual se
centrará el presente informe con el fin de conocer y analizar las
variables climáticas que allí se presentan.

La altura a la que se puede encontrar un páramo no es igual para todos
los casos, pues el límite inferior de estos es variable según la
latitud, la vertiente, el clima global y la actividad humana. En América
se encuentran entre los \(3000\) y \(4800 \ msnm\) aproximadamente, para
Colombia, en las cordilleras central y occidental está a \(3500 \ msnm\)
y en la oriental a \(3600 \ msnm\). La zonificación típica utilizada en
la alta montaña colombiana corresponde a bosque alto andino (\(3000\) a
\(3200 \ msnm\)), páramo bajo o subpáramo (entre \(3200\) y \(3500\) o
\(3600 \ msnm\)), páramo propiamente dicho (entre \(3500\) o \(3600\) y
\(4100 \ msnm\)) y superpáramo (entre \(4100\) y \(4500 \ msnm\))
\cite{ortizparamos}. Diferentes autores confirman que el clima en los
páramos realmente es muy variado, aunque se presenten condiciones de
altura similares y proximidad \cite{paramos}, esta variabilidad se
presenta en todas las características climáticas, tales como
precipitación, temperatura, radiación, velocidad del viento y humedad
relativa, y aunque hay todavía pocas estaciones climáticas en todos los
páramos es evidente la variación en los resultados de la medición de
estos parámetros climáticos.

Por lo general en la transición de bosque y el subpáramo las
temperaturas medias multianuales en algunos caso pueden ser incluso
menores a \(9^{\circ}C\), aproximadamente por encima de los
\(3300 \ msnm\), en el páramo medio podrían llegar a ser menores de
\(6^{\circ}C\) y ya en el superpáramo cerca de las nieves perpetuas son
inferiores a \(3^{\circ}C\) \cite{morales2019atlas}. En cuanto a la
variación de la temperatura media mensual no hay grandes cambios, sin
embargo en los páramos la temperatura puede variar a gran escala durante
el día y la noche. En la precipitación hay una amplio rango y un gran
contraste entre los páramos de Colombia, la precipitación puede variar
entre los \(700\) y \(5000 \ mm\) al año, algunos de los páramos tienen
un régimen de lluvias monomodal como el páramo chingaza
\cite{morales2019atlas} y otros bimodal como el complejo Guantiva - La
Rusia \cite{morales2019atlas}; los páramos secos se encuentran en las
vertientes oriental de la cordillera oriental y occidental de la
cordillera occidental, en cuanto a los más secos se encuentran entre la
cordillera oriental \cite{morales2019atlas}. Los ecosistemas de páramo
presenta una humedad relativa alta que es variable y estacional siendo
máxima en tiempos de lluvia y mínima en tiempos secos, usualmente en un
rango que comprende entre un \(80\) y \(90 \%\) esto debido a un factor
de suma importancia en los páramos como lo es el fenómeno de niebla
\cite{morales2019atlas}. Comúnmente la evapotranspiración en los páramos
es baja pues casi siempre el ambiente es muy cercano a la saturación y
se presenta un alta radiación ultravioleta sobre todo en periodos secos
y abundancia de luz difusa \cite{morales2019atlas}. Por último los
vientos en los páramos son muy variables pero regularmente los más
intensos se dan en los páramos que se encuentran en las vertientes de
los valles interandinos \cite{morales2019atlas}.

\hypertarget{materiales-y-muxe9todos}{%
\section{MATERIALES Y MÉTODOS}\label{materiales-y-muxe9todos}}

\hypertarget{localizaciuxf3n-y-descripciuxf3n-del-uxe1rea-de-estudio}{%
\subsection{Localización y descripción del área de
estudio}\label{localizaciuxf3n-y-descripciuxf3n-del-uxe1rea-de-estudio}}

El páramo la Rusia se encuentra ubicado en límites de los departamentos
de Boyacá y Santander, en el flanco occidental de la cordillera
oriental, entre los \(3100\) y \(4280 \ msmn\). Este páramo hace parte
de un extenso corredor de páramos y bosques alto andinos denominado como
Guantiva - La Rusia, complejo que incluye a los páramos de Cruz
Colorada, Guina, Pan de Azúcar, Carnicerías y Guata y que tiene una
extensión en área de \(119.009 \ ha\) (Corpoboyacá y CAS, 2017), en el
cual predomina una topografía abrupta que varía de acuerdo con la
alternancia de las formaciones geológicas presentes. El páramo está
influenciado por la Zona de Convergencia Intertropical (ZCIT) y el
movimiento de las corrientes de vientos locales e inter tropicales, lo
que genera un régimen húmedo. El régimen de lluvias es bimodal, con una
precipitación máxima entre abril, mayo, octubre y noviembre (UPTC, 2015)

El sitio de estudio se encuentra dentro del páramo La Rusia en la vereda
San alejo, municipio de Duitama Boyacá; contiguo por el norte a los
límites con el municipio de Charalá departamento de Santander con
coordenadas \(5^{\circ}57’48.0"N\) \(73^{\circ}05'16.3"W\) y una altitud
mayor a los \(3500\ msnm\).

\hypertarget{levantamiento-de-informaciuxf3n}{%
\subsection{Levantamiento de
información}\label{levantamiento-de-informaciuxf3n}}

Dentro del páramo de La Rusia se instalaron varias estaciones
climatológicas cercadas las cuales midieron temperatura, humedad
relativa, velocidad y dirección del viento, precipitación y radiación
solar. Para este artículo sólo se tuvo en cuenta una estación
climatológica con datos para 6 meses del año 2019, de enero a junio,
tomados cada 15 minutos para todas las variables mencionadas. La
estación climática de la cual se obtuvieron los datos hace parte de un
complejo de estaciones climáticas que ha venido instalando el equipo de
trabajo del Dr.~Mark Mulligan de Kings College, como parte de un
proyecto para probar y calibrar equipos de monitoreo open source más
rentables para los proyectos investigativos.

También se usaron dos sensores HOBO móviles para recolectar información
de temperatura y humedad relativa en tiempo real con intervalos de 1
minuto, los cuales fueron puestos en funcionamiento a partir del día
jueves 12 de marzo hasta el sábado 14 de marzo de las 8am a las 5pm y
para el día domingo entre las 8am y las 3pm, llevando los sensores por
parte del complejo de páramos Guantiva- La Rusia a la vez que se iban
instalando las estaciones climáticas y se tomaban otros datos como
caudal y muestras de suelos con el resto del grupo de trabajo, datos que
no se tendrán en cuenta en este informe. Así mismo el uso del software
de la misma empresa (HOBO) para dispositivos celulares en el cual se
podían verificar los datos en tiempo real.

\hypertarget{procesamiento-de-los-datos-recolectados}{%
\subsection{Procesamiento de los datos
recolectados}\label{procesamiento-de-los-datos-recolectados}}

Los datos climáticos pueden proporcionar una gran cantidad de
información sobre el medio ambiente atmosférico que afecta a casi todos
los aspectos del esfuerzo humano \cite{Bala}, es por ello que es
importante el análisis estos, para determinar tendencias en las
variables que se puedan interpretar buscando entender el comportamiento
y así tomar decisiones que más convengan. Buscando el filtrado y
análisis de los datos se utilizó \texttt{R\ versión\ 3.6.1}. Para los
datos de precipitación se usó la suma de los los valores diarios por mes
y para los demás (temperatura, humedad relativa y velocidad del viento)
los valores promedio. Se graficaron las variables por separado buscando
propensiones para la descripción de cada una de ellas, luego se buscaron
relaciones estadísticas entre variables con el fin de determinar acaso
alguna dependencia entre los datos. Con los datos de temperatura del
sensor HOBO se realizó un modelo kriging ordinario de primer orden y
transformación logarítmica en el software Arcgis versión 10.5 esto con
el fin de observar un comportamiento aproximado de la variable.

\hypertarget{resultados-y-discusiuxf3n}{%
\section{Resultados y discusión}\label{resultados-y-discusiuxf3n}}

En la \textbf{Fig.1} se puede observar la precipitación mensual
discriminada por la cantidad de precipitación de cada día (representados
en colores), en esta escala no se percibe todo el ciclo anual pero si la
temporada de lluvias entre marzo y mayo y parte de una temporada seca a
la que se debe la baja precipitación en febrero, lo que responde al
régimen bimodal del páramo La Rusia. La precipitación total de estos 6
meses suma \(1096.2mm\), una precipitación alta, característica de las
laderas orientadas al occidente, pues las laderas internas de los Andes
están altamente influenciadas por efectos de sombra de lluvia, para las
lluvias que llegan tanto desde la cuenca del Amazonas como de la costa
del Pacífico \cite{buytaert2006hidrologia}.

\begin{figure}
\centering
\includegraphics{Hidrology_files/figure-latex/unnamed-chunk-1-1.pdf}
\caption{Tendencia Diaría de precipitación}
\end{figure}

En la \textbf{Fig.2}, se observa la variación de la temperatura
alrededor del promedio de los seis meses de \(9.2^{\circ}C\) con una
temperatura máxima de \(19.5^{\circ}C\) y una mínima de
\(1.6^{\circ}C\), sin embargo la fluctuación de mayor densidad se
encuentra entre los \(8^{\circ}C\) y \(10^{\circ}C\). La variación
diurna de la temperatura resulta del ciclo de insolación superficial
\cite{poveda2004hidroclimatologia}, la cual es mayor entre las 11:00
a.m. y la 1:00 p.m. Debido a su localización cercana a la línea
ecuatorial, la radiación solar diaria es casi constante todo el año.
Esta constancia resulta en una baja variabilidad estacional en
temperatura media del aire, en contraste con el ciclo diario, el cual es
totalmente marcado \cite{buytaert2006hidrologia}; en la \textbf{Fig.3}
se muestra distribución de la temperatura a lo largo del periodo de
medición, es notable una \texttt{normal}, de esto es fácil deducir que
los valores tienden a un valor medio, es decir, que generalmente los
valores de temperatura estarán en el rango de \(9 a 10^{\circ}\), hay
ocasiones como en el mes 1 en donde la variable tiende a un valor más
bajo de temperatura.

\begin{figure}
\centering
\includegraphics{Hidrology_files/figure-latex/unnamed-chunk-5-1.pdf}
\caption{Distribución de la temperatura}
\end{figure}

\begin{figure}
\centering
\includegraphics{Hidrology_files/figure-latex/unnamed-chunk-6-1.pdf}
\caption{Tendencia de la temperatura mensual}
\end{figure}

En la \textbf{Fig.4} se presenta la distribución de la humedad relativa,
esta es parecida a una \texttt{Fisher} sin embargo no tiene relación con
el caso pues no se está comparando poblaciońes, más bien parece haber la
tendencia a que en el páramo haya una alta humedad relativa centrada en
valores \(> \ 90 \%\) lo cual es relacionable con la variable anterior
pues a mayor temperatura del aire hay mayor capacidad en retener humedad
\cite{jaramillo}, se puede inferir entonces que la temperatura del
páramo es fría solo con mirar su humedad relativa, este comportamiento
está presente en los datos pues hay un valor bajo de temperatura con un
alto porcentaje de humedad.

\begin{figure}
\centering
\includegraphics{Hidrology_files/figure-latex/unnamed-chunk-8-1.pdf}
\caption{Distribución de la Humedad relativa}
\end{figure}

\begin{figure}
\centering
\includegraphics{Hidrology_files/figure-latex/unnamed-chunk-11-1.pdf}
\caption{Distribución de la velocidad del viento}
\end{figure}

En la \textbf{Fig.5} se muestra la distribución de la velocidad del
viento, los valores medios oscilan entre \(50\) y \(100 \frac{m}{s}\),
este parámetro es relacionado con la precipitación pues si hay grandes
vientos estos pueden desplazar las masas de aire disminuyendo la
probabilidad del evento \cite{tobon}, por ejemplo, el mes 2 fue el que
tuvo menor precipitación la velocidad del viento para esa fecha es
variable lo cual no explicaría la baja precipitación en ese mes, sin
embargo, al mirar la \textbf{Fig.6} distribución de la radiación solar,
es fácil ver como en la fecha el sol no fue constante lo que favoreció a
su vez que la humedad relativa presenta los valores más pequeños en todo
el registro, las masas de aire entonces se calientan cuando hay una
temperatura mayor por ende más radiación solar, estas ascienden y la
velocidad del viento puede o no llevarlas a otro lugar controlando así
la precipitación en el sitio \cite{tobon}.

\begin{figure}
\centering
\includegraphics{Hidrology_files/figure-latex/unnamed-chunk-14-1.pdf}
\caption{Distribución de la radiación solar media}
\end{figure}

En la \textbf{Fig.7} se puede observar la correlación para las distintas
variables climáticas, presentando así una correlación positiva entre la
precipitación y humedad relativa (\(0.38\)), esto puede darse ya que en
épocas de lluvia la humedad relativa es constantemente alta y tiende a
la saturación en eventos de precipitación y además suele presentarse el
fenómeno de niebla. \cite{morales2019atlas}, dato que se puede
corroborar en la \textbf{Fig.10} que muestra la relación entre
precipitación y humedad relativa, que presenta una línea de tendencia
con un aumento muy rápido en la humedad relativa mientras inicia la
precipitación y luego se mantiene constante en el evento de lluvia
tendiendo a la saturación del ambiente. Lo mismo ocurre con las
variables de radiación solar y temperatura que tienen una correlación
positiva de \(0.52\), la cual puede darse por el gran aumento de
insolación solar y temperatura que se presenta a medio día en el páramo
ya que se encuentra muy cerca de la línea ecuatorial y recibe una gran
radiación diaria todo el año mientras se tenga un cielo despejado
\cite{buytaert2006hidrologia}. Esto se puede ver en la \textbf{Fig.9}
que muestra la relación entre temperatura y radiación solar, donde la
línea de tendencia muestra una relación directa en el aumento de la
radiación y temperatura. Caso contrario cuando se analiza la correlación
entre radiación solar y humedad relativa (\(-0.81\)) o temperatura y
humedad relativa (\(-0.25\)), obteniendo valores negativos; esto se pudo
evidenciar en campo, pues mientras la temperatura era más alta el aire
se sentía mucho más seco. Como se puede ver en la \textbf{Fig.8} que
muestra la relación entre temperatura vs humedad relativa, donde la
línea de tendencia disminuye a medida que la temperatura aumenta.

\begin{figure}
\centering
\includegraphics{Hidrology_files/figure-latex/unnamed-chunk-18-1.pdf}
\caption{Matrix de correlación para las variables climáticas}
\end{figure}

\begin{figure}
\centering
\includegraphics{Hidrology_files/figure-latex/unnamed-chunk-19-1.pdf}
\caption{Temperatura vs Humedad relativa}
\end{figure}

\begin{figure}
\centering
\includegraphics{Hidrology_files/figure-latex/unnamed-chunk-20-1.pdf}
\caption{Temperatura vs Radiación solar}
\end{figure}

\begin{figure}
\centering
\includegraphics{Hidrology_files/figure-latex/unnamed-chunk-21-1.pdf}
\caption{Precipitación vs Humedad relativa}
\end{figure}

\begin{figure}
\centering
\includegraphics{Hidrology_files/figure-latex/unnamed-chunk-22-1.pdf}
\caption{Velocidad del viento vs Temperatura}
\end{figure}

En la \textbf{Fig.8} se puede observar la reacción que se encontró para
los parámetros climáticos temperatura y humedad relativa durante el
periodo del que se tienen los datos. La gráfica de dispersión ilustra un
comportamiento inverso entre las dos variables con una línea de
tendencia que en general es decreciente, a medida que comienza a
ascender la temperatura la humedad relativa comienza a descender lo que
es normal pues a medida que el ambiente se torna más caliente, el aire
se torna más seco lo que en un páramo está sujeto a la estacionalidad,
pues la humedad relativa es variable y estacional \cite{hofstede2017p},
así en épocas de lluvia habrá mayor humedad relativa que en épocas secas
o de verano, la variación de este factor está estrechamente ligada a los
fenómenos de niebla que en un páramo pueden presentarse con mayor o
menor frecuencia dentro de un periodo de tiempo. En síntesis la gráfica
no se sale del comportamiento normal de estos dos parámetros
climatológicos (humedad relativa y temperatura), debido a que estos son
normalmente inversos, cabe destacar que la humedad relativa extrañamente
baja a valores menores de 70 lo que es característico de estos
ecosistemas. \cite{hofstede2017p}

\begin{figure}
\centering
\includegraphics{Hidrology_files/figure-latex/unnamed-chunk-23-1.pdf}
\caption{Distribución empírica acumulada de temperatura}
\end{figure}

En la \textbf{Fig.9} se observa la relación entre la temperatura y la
radiación solar, de la gráfica se deduce que estos dos parámetros son
casi siempre directamente proporcionales pues en la mayoría del tiempo a
medida que la temperatura aumenta la radiación solar comienza también a
aumentar, y es que esto mientras se tengan las condiciones necesarias
(como un cielo despejado) es normal en un páramo pues debido a su
altitud y cercanía con el ecuador la radiación solar que reciben estos
ecosistemas es alta mientras no haya nubosidad
\cite{montenegro2015estimacion}.

En la \textbf{Fig.12} se observa la distribución empírica acumulada de
la temperatura, es evidente notar que la probabilidad de encontrar
temperaturas menores a \(11^{\circ}C\) es alta, lo contrario pasa con
valores muy bajos del parámetro, es poco probable encontrar valores
cercanos a cero de hecho los valores más comunes giran en torno al rango
de \(8\) a \(11^{\circ}C\).

En la \textbf{Fig.13} se muestra el modelo de temperatura construido a
partir de los datos del sensor HOBO, cabe la pena aclarar que este es un
aproximación y no es un modelo que ajuste bien los datos, es válido
decir esto dada la cantidad de puntos utilizados (\(8\)), son pocos,
pues se tomaron las coordenadas de los instrumentos instalados. Se puede
notar el gradiente mostrado en el mapa, sugiere la variabilidad de la
temperatura en el páramo; los rangos mostrados difieren en
aproximadamente \(3^{\circ}C\), al mirar este comportamiento, se
procedió a verificar si la altura tenía influencia en la temperatura, lo
esperado sería que este parámetro tuviera influencia \cite{basantes},
para determinar la relación se hace un test de correlación arrojando un
resultado de \(-0.46\) lo cual indica que mientras una variable aumenta
la otra disminuye, sin embargo, el mismo valor en sí es deja enduda si
es una relación lineal, pues se encontró un \(R^2\) de \(0.22\), lo que
demuestra que si bien hay relación en entre los parámetros esta puede
fluctuar y no ser constante, es decir, pueden haber lugares altos pero
con temperaturas más altas de lo normal, este comportamiento no sigue
descrito por \cite{van} la tasa de cambio en el promedio de temperatura
con respecto a la altitud, está típicamente entre \(0.6\) y
\(0.7^{\circ}C\) \(100 \ m-1\), esto tiene explicación pues la subida de
temperatura, causada por el efecto invernadero de varios gases
antropogénicos de los cuales el CO2 es el más conocido, es el proceso
fundamental global del cambio climático, esto viene acompañado de otros
efectos secundarios \cite{buyta} los cuales no serán tratados acá.

\begin{figure}
\centering
\includegraphics{paramo3.png}
\caption{Modelo de temperatura}
\end{figure}

\hypertarget{conclusiones}{%
\section{Conclusiones}\label{conclusiones}}

Las variables climáticas están íntimamente relacionadas, una depende la
otra como un ciclo que necesita de su entorno, las influencias pueden
variar de manera positiva o negativa, es decir, que si una aumenta la
otra disminuye o viceversa, es urgente entonces prestar atención a
fenómenos como el cambio climático que modifica algunas variables del
ambiente, esto repercute en el entorno alterando las fases naturales y
llevando consigo la disminución de la calidad de vida para todos los
seres vivos.

El clima en el páramo de La Rusia presenta la variabilidad esperada con
su temperatura máxima de \(19.5^{\circ}C\) a medio día y una mínima de
\(1.6^{\circ}C\) en la madrugada, una humedad relativa constantemente
alta en excepción del tiempo donde se tiene la temperatura más elevada
haciendo de este un páramo muy húmedo, que si se tiene una buena
regeneración del ecosistema con unos suelos ricos en porosidad y óptimos
en infiltración, ayudado de la vegetación, puede ser muy importante para
la captura de agua y alimentación de los acuíferos subterráneos y ríos,
proporcionando así una buena oferta hídrica. Por lo que se hace
necesario estudios más detallados para la conservación de estos
ecosistemas sobre todo en un país como Colombia que tiene la mitad de
páramos del mundo en el cual cae la responsabilidad en sus ciudadanos
para un manejo óptimo y sostenible.

\bibliography{mybibfile}

\end{document}


